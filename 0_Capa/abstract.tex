%-------------------------------------------------------------------
% -- Abstract %-------------------------------------------------------------------
\chapter*{Abstract}
\thispagestyle{empty}


This work presents a low cost remote monitoring and control system
for air conditioners, which uses Internet of Things concepts. The main tool used was the ESP8266 microcontroller, which through the MQTT protocol receives and returns requests from a mobile application over the internet. A presence sensor was used to alert unnecessary power consumption, electric drive devices to power the load and an electric energy meter circuit was developed to inform the air conditioner consumption. In addition, the Network Time Protocol was used to synchronize the clock of the microprocessor, in this way it was possible to turn on and off the conditioner according to the chosen time in the application. Finally, an experimental comparative of the electric power consumption measurement between the developed prototype and a energy meter already distributed in the market is shown, in order to validate the system for deployment.

\vspace*{\stretch{1}} %texto se adapta para ocupar toda a pagina

\noindent \textsf{Keywords:} Internet of Things, Air Conditioner, ESP8266, MQTT, Remote Control.

\cleardoublepage
