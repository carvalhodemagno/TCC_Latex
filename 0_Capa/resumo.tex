%-------------------------------------------------------------------
% -- Resumo %-------------------------------------------------------------------
\chapter*{Resumo}
\thispagestyle{empty}
Este trabalho apresenta um sistema de monitoramento e controle remoto de baixo custo para condicionadores de ar, que
utiliza conceitos de Internet das Coisas. A principal ferramenta utilizada foi o microcontrolador ESP8266, que atrav�s do protocolo MQTT recebe e retorna requisi��es de um aplicativo \textit{mobile} pela internet. Foram utilizados um sensor de presen�a, para
alertar o consumo desnecess�rio de energia, dispositivos de acionamento el�trico para acionar a carga e foi desenvolvido um circuito medidor de energia el�trica para informar o consumo do condicionador de ar. Al�m disso, foi utilizado o protocolo \textit{Network Time Protocol} para sincronizar o
rel�gio do microprocessador, desta maneira foi poss�vel ligar e desligar o condicionador de ar conforme hor�rio escolhido no aplicativo. Por fim � apresentado um comparativo experimental da medi��o do consumo de energia el�trica entre o prot�tipo desenvolvido e um medidor de energia j� difundido no mercado, de modo a validar o sistema para implanta��o.

\vspace*{\stretch{1}} %texto se adapta para ocupar toda a p�gina

\noindent \textsf{Palavras-chave:} Internet das coisas, condicionador de ar, ESP8266, MQTT, controle remoto.

\cleardoublepage
