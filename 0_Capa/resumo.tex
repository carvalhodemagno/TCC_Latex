%-------------------------------------------------------------------
% -- Resumo %-------------------------------------------------------------------
\chapter*{Resumo}
\thispagestyle{empty}
Este trabalho apresenta um sistema de monitoramento e controle remoto para condicionadores de ar de baixo custo, que
utiliza conceitos de Internet das Coisas. A principal ferramenta utilizada foi o microcontrolador ESP8266, que atrav�s do protocolo MQTT recebe e retorna requisi��es de um aplicativo \textit{mobile}. Foram utilizados um sensor de presen�a, para
alertar o consumo desnecess�rio de energia, dispositivos de acionamento el�trico para acionar a carga e um medidor de energia el�trica foi desenvolvido para informar o consumo do condicionador de ar. Foi utilizado o protocolo Network Time Protocol para sincronizar o
rel�gio do microprocessador, desta maneira foi poss�vel ligar e desligar o condicionador de ar conforme hor�rio escolhido no aplicativo. Para a programa��o do microcontrolador
foi utilizada a linguagem de programa��o C++ juntamente com o software Arduino,
que cont�m as bibliotecas de fun��es necess�rias para a realiza��o do trabalho. 

\vspace*{\stretch{1}} %texto se adapta para ocupar toda a p�gina

\noindent \textsf{Palavras-chave:} Internet das coisas, condicionador de ar, ESP8266, MQTT, controle remoto.

\cleardoublepage
