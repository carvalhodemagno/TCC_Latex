%%%%%%%%%%%%%%%%%%%%%%%%%%%%%%%%%%%%%%%%%%%%%%
\chapter{Fundamenta��o Te�rica}
\label{chapter:fundamentacao_teorica}
%%%%%%%%%%%%%%%%%%%%%%%%%%%%%%%%%%%%%%%%%%%%%%
Este cap�tulo apresenta os fundamentos te�ricos utilizados para o desenvolvimento deste trabalho. Primeiramente, a se��o \ref{sec:iot_mqtt} explanar� a Internet das Coisas (IoT) e um dos protocolos mais utilizados por esta tecnologia, o \textit{Message Queuing Telemetry Transport} (MQTT), que � um dos protocolos mais utilizados quando se trata de comunica��o neste tipo de rede. Em seguida, na se��o \ref{sec:dispositivos_acionamento_eletrico}, ser�o abordados os dispositivos utilizados quando se trata de acionamentos el�tricos, dando enfoque aos respons�veis pelo acionamento de cargas do tipo resistiva e indutiva. Por final ser� abordado, na se��o \ref{sec:medicao_energia_eletrica}, a medi��o de energia el�trica e os Circuitos Integrados de Gerenciamento de Energia (PMICs), enfatizando suas utiliza��es em sistemas de monitoramento da qualidade da energia el�trica.

\section{Internet das Coisas e o Protocolo MQTT}
\label{sec:iot_mqtt}
O termo Internet das Coisas (IoT) surgiu quando um grupo do \textit{Massachusetts Institute of Technology} (MIT) trabalhava no campo de identifica��o, localiza��o e reconhecimento de estado de objetos usando sensores sem fio e tecnologia de identifica��o por radiofrequ�ncia \cite{rise_of_iot}.
 
A IoT � um servi�o da Internet que permite dispositivos f�sicos a comunicarem entre si ou com pessoas, atrav�s da rede mundial de computadores, possibilitando assim adquirir e monitorar informa��es. 

Com o trabalho colaborativo de v�rios sistema, a IoT torna ambientes mais inteligentes, permitindo repostas autom�ticas - ou seja, sem interfer�ncia humana - a dados adquiridos do ambiente, como monitoramento de cheias, informa��es sobre terremotos, dados sobre o tr�fego, entre outros \cite{iot_footbal, iot_open_data}. 


\section{Dispositivos de Acionamento El�trico}
\label{sec:dispositivos_acionamento_eletrico}


\section{Medi��o de Energia El�trica e os PMICs}
\label{sec:medicao_energia_eletrica}


